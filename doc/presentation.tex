\documentclass[10pt, handout]{beamer}

\usetheme[progressbar=frametitle]{metropolis}
\usepackage{appendixnumberbeamer}

\usepackage[italian]{babel}
\usepackage{booktabs}
\usepackage[scale=2]{ccicons}
\usepackage{float}

\usepackage{tikz}
\usetikzlibrary{calc}
\usetikzlibrary{quotes}
\usetikzlibrary{patterns}
\usetikzlibrary{angles}

\usepackage[european]{circuitikz}
\usepackage{tikz-timing}
\usepackage{tikzscale}

\usepackage{pgfplots}
\usepgfplotslibrary{dateplot}

\usepackage{xspace}

\title{Spectrum Analyzer}
\subtitle{Lavoro Professionale Individuale}
\date{\today}
\author{Naoki Pross}
\institute{SAM Bellinzona}
\titlegraphic{\hfill\includegraphics[height=1.3cm]{figures/logo/LOGO_SAM.pdf}}

\begin{document}

\maketitle

\begin{frame}{Table of contents}
    \setbeamertemplate{section in toc}[sections numbered]
    \tableofcontents[hideallsubsections]
\end{frame}

\section{Introduzione}
\begin{frame}{Obiettivo}
    Realizzare un circuito di analisi spettrale
    \begin{block}{Requisiti}
    \begin{itemize}
        \item Analisi dello spettro fino a 10\,kHz
        \item Entrate Jack e RCA
        \item Visualizzazione 
        \item Utilizzo di un PIC18F45K22
    \end{itemize}
    \end{block}
    
    \pause
    \begin{block}{Componenti}
    \begin{itemize}
        \item Circuito di adattamento in entrata
        \item Design di un PCB
        \item Software per il uC e per il PC
    \end{itemize}
    \end{block}
\end{frame}

\begin{frame}{Schema a blocchi}
    \resizebox{\textwidth}{!}{
        \includegraphics{figures/block-diagram.tikz}
    }
\end{frame}

\section{Fourier Transform}
\begin{frame}{Rappresentazione grafica}
\end{frame}

\section{Fast Fourier Transform}
\begin{frame}{Il problema della DFT}
\end{frame}

\begin{frame}{}
\end{frame}

\section{Prodotto realizzato}
%\begin{frame}{}
%\end{frame}

\section{Conclusioni}
\begin{frame}{Obiettivi}
    \begin{block}{Raggiungi}
    \begin{itemize}
        \item Analisi spettrale fino a 10\,kHz
        \item Visualizzazione al PC\(^\dagger\)
        \item Esportare immagini / dati
    \end{itemize}
    \end{block}
    \pause

    \begin{block}{Incompleti}
    \begin{itemize}
        \item \(^\dagger\) Visualizzazione delle curve \(\Re(z)\) e \(\Im(z)\) in un solo grafico
        \item Visualizzazione mediante la matrice LED
    \end{itemize}
    \end{block}
\end{frame}

\begin{frame}{Possibili miglioramenti}
\end{frame}


\end{document}
