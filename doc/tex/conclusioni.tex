\chapter{Conclusioni}
\section{Risultati}

\section{Problemi riscontrati}
\subsection{Errore nella scelta dell'opamp}
\label{sec:err-opamp}

Durante la fase di test, dopo l'assemblaggio, si \`e notato che
l'amplificatore operazionale AD826 non \`e un opamp rail to rail come l'AD820,
utilizzato durante la fase di sviluppo su piastra sperimentale.  Ci\`o limita
l'escursione del segnale amplificato di quasi 1\,V, riducendo la precisione
del campionamento.

Fortunatamente il pinout DIP8 degli operazionali dual package \`e standard,
perci\`o non vi sono eccessive difficolt\`a nella sostituzione del componente.
Purtroppo per\`o non sar\`a possibile acquistare in tempo un opamp
sostitutivo.

\subsection{Errore nel dimensionamento del filtro attivo}
\label{sec:err-filter}

Durante la fase di test \`e stato inoltre rilevato che il filtro attivo (vedi
figura \ref{fig:filter-ampl}) di anti-alias aveva un amplificazione non
unitaria, dunque incorretta.  La tensione di offset di 2.5\,V veniva
amplificata ed in molti punti l'operazionale entrava in saturazione.  Ci\`o
era causato dal valore incorretto di \(R_{11}\) poich\`e l'amplificazione,
data dalla relazione sottostante, era  di 2.
\[
    A_v = 1+R_{12}/R_{11} = 1+15\,\text{k}\Omega/15\,\text{k}\Omega = 2
\]

Il valore corretto per il resistore \(R_{11}\) \`e
\(\geq 750\,\text{k}\Omega\), in modo da ottenere un amplificazione quasi
unitaria, con un errore di \(+0.02\), che corrisponde ad un incremento di
50\,mV della tensione di offset di 2.5\,V. Valori di ordini di grandezza
maggiori sono preferibili, infatti dopo aver rilevato il guasto a \(R_{11}\)
\`e stato assegnato un valore di \(910\,\text{k}\Omega\).
\[
    A_v = 1+R_{12}/R_{11} = 1+15\,\text{k}\Omega/910\,\text{k}\Omega \approx 1.016
\]

\subsection{Sincronizzazione dei threads}
\label{sec:err-sync}
Durante lo sviluppo del software desktop \`e stato riscontrato un unico
problema riguardante la sincronizzazione dei threads. La risorsa
\texttt{serial::Serial MainWindow::\_serial}  come implica il nome \`e
instanziata nella classe \texttt{MainWindow}, ma essa \`e gestita anche dalla
classe \texttt{SerialWorker} siccome \`e suo compito leggere i dati.

Perci\`o la risorsa deve essere protetta da un \texttt{QMutex} e la sua
\emph{lifetime} (ciclo di vita) deve essere gestita tenendo in considerazione
il thread parallelo.  Il bug era causato da una chiusura della risorsa seriale
mentre il thread era ancora attivo.  Chiudendo la risorsa mentre il thread del
\texttt{SerialWorker} \`e attivo, al prossimo tentativo di lettura il metodo
\texttt{serial::Serial::read()} causa una \texttt{serial::IOException} che fa
crashare il programma.

La ragione per cui il thread non veniva fermato, era da da utilizzo incorretto
dell'API dei \texttt{QThread}. Per chiudere correttamente un thread secondo il
framework di Qt si richiede un interruzione con
\texttt{QThread::requestInterruption()}. Invece nel codice veniva utilizzato
\texttt{QThread::quit()}, che se non in condizioni particolari non chiude il
processo parallelo.

Il diff  sottostante mostra il commit in cui il problema viene risolto.
\lstinputlisting[language=diff]{res/serialworker-crash-fix.diff}

\section{Commento}
Personalmente ho trovato il progetto molto interessante e coinvolgente.
Malgrado la complessit\`a dell'argomento trattato, grazie al supporto di
docenti ed amici, sono riuscito ad avere una comprensione tutto sommato
abbastanza completa del funzionamento del principio matematico dell'analisi
spettrale.

\section{Ringraziamenti}
Vorrei ringraziare Eduardo Cima: professore di elettrotecnica alla SAM e
Raffaele Ancarola: studente di fisica del primo anno al Politecnico Federale
di Losanna (EPFL), per il grande supporto attraverso spiegazioni e chiarimenti
degli strumenti matematici della trasformata di Fourier; ed infine vorrei
ringraziare anche il professor Emidio Planamente per l'aiuto a risolvere il
bug di sincronizzazione (\ref{sec:err-sync}).


\section{Certificazione}
Il sottoscritto dichiara di aver redatto e prodotto individualmente il lavoro
di produzione.
\begin{flushright}
\begin{tabular}{ r p{5cm} p{1cm} r p{5cm}}
    Data: & \hrulefill && Firma: & \hrulefill \\
    &&&& Naoki Pross \\
\end{tabular}
\end{flushright}
