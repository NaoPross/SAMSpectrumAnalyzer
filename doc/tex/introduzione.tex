\chapter{Introduzione}

\section{Contesto}
Per portare a termine il percorso formativo per un attestato di capacit\`a
federale presso la Scuola Arti e Mestieri di Bellinzona \`e richiesto lo
sviluppo individuale di un progetto di produzione di un prodotto.
Per interesse personale nella matematica della trasformata di Fourier mi \`e
stato assegnato di sviluppare un analizzatore spettrale.

\section{Requisiti}
\`E richiesto di sviluppare circuito per analizzare lo spettro dei segnali di
frequenza fino a 10\,kHz. Il dispositivo dovr\`a avere 3 possibili sorgenti:
RCA/Cinch e 2 Audio Jack per un microfono e per una sorgente di audio
generica. \`E inoltre richiesto che il calcolo dei dati dello spettrogramma
sia eseguito da un microcontroller della Microchip, collegato a due
altri dispositivi quali, una display e ad un computer in RS232, per poter
visualizzare lo spettrogramma computato.

\section{Concetti matematici}
Il circuito realizzato si appoggia sul concetto matematico di importanza
fondamentale nelle discipline come la fisica e l'elettrotecnica della
\emph{Trasformata di Fourier}. Questa operazione matematica \`e fondata su su
un principio dimostrato da Joseph Fourier che asserisce che \`e possibile
rappresentare una qualsiasi funzione periodica, in alcuni casi anche non
periodica, con una serie di sinusoidi di frequenze multiple ad una di base.
L'operazione di \emph{Trasformata} dunque \`e uno strumento per osservare
le frequenze di queste armoniche, esso trasforma una funzione in funzione del
tempo \(f(t)\) in una funzione rispetto alla frequenza o alla pulsazione
\(\hat f(\omega)\).

Secondariamente, il progetto usufruisce anche di un altro strumento chiamato
\emph{Fast Fourier Transform} (FFT) scoperto inizialmente nel 1965 dai
matematici J. Cooley e J. Tukey. La FFT \`e un algoritmo con molte
implementazioni che riduce la complessit\`a computazionale della trasformata
di fourier discreta da \(\mathcal{O}(n^2)\) a \(\mathcal{O}(n \log n)\).
Questo \`e necessario perch\`e le operazioni matematiche da eseguire sono dei
prodotti tra numeri complessi, i quali causerebbero dei severi cali di
performance.

Tutti i concetti descritti saranno approfonditi nei capitoli seguenti.

\section{Norme di progetto}

