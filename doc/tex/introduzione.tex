\chapter{Introduzione}

\section{Contesto}
Per portare a termine il percorso formativo per un attestato di capacit\`a
federale presso la Scuola Arti e Mestieri di Bellinzona \`e richiesto lo
sviluppo individuale di un progetto di produzione di un prodotto.
Per interesse personale nella matematica della trasformata di Fourier mi \`e
stato assegnato di sviluppare un analizzatore spettrale.

\section{Requisiti}
\`E richiesto di sviluppare circuito per analizzare lo spettro dei segnali di
frequenza fino a 10\,kHz. Il dispositivo dovr\`a avere 3 possibili sorgenti:
RCA/Cinch e 2 Audio Jack per un microfono e per una sorgente di audio
generica. \`E inoltre richiesto che il calcolo dei dati dello spettrogramma
sia eseguito da un microcontroller della Microchip, collegato a due
altri dispositivi quali, un display e ad un computer in RS232, per poter
visualizzare lo spettrogramma computato.

\section{Concetti matematici}
Il circuito realizzato si appoggia sul concetto matematico di importanza
fondamentale, nelle discipline come la fisica e l'elettrotecnica della
\emph{Trasformata di Fourier}. Questa operazione matematica \`e fondata su
un principio dimostrato da Joseph Fourier che asserisce che \`e possibile
rappresentare una qualsiasi funzione periodica, in alcuni casi anche non
periodica, con una serie di sinusoidi di frequenze multiple ad una di base.
L'operazione di \emph{Trasformata} dunque \`e uno strumento per osservare
le frequenze di queste armoniche, esso trasforma una funzione in funzione del
tempo \(f(t)\) in una funzione rispetto alla frequenza o alla pulsazione
\(\hat f(\omega)\), che restituisce ad ogni \(\omega\) l'ampiezza e la fase
dell'armonica.

Secondariamente, il progetto usufruisce anche di un altro strumento chiamato
\emph{Fast Fourier Transform} (FFT) scoperto inizialmente nel 1965 dai
matematici J. Cooley e J. Tukey. La FFT \`e un algoritmo con molte
implementazioni che riduce la complessit\`a computazionale della trasformata
di fourier discreta da \(\mathcal{O}(n^2)\) a \(\mathcal{O}(n \log n)\).
Questo \`e necessario perch\'e le operazioni matematiche da eseguire sono dei
prodotti tra numeri complessi, i quali impiegano molto tempo per essere
computati.

Tutti i concetti descritti saranno approfonditi nei capitoli seguenti.

\section{Norme di progetto}
%\subsection{Software}
{\renewcommand\arraystretch{1.2}
\begin{table}[H] \centering
    \caption{Norme di progetto: Software}
    \begin{tabularx}{\textwidth}{X l}
        \toprule
        \bfseries Componente & \bfseries Software \\
        \midrule
        Version control  & Git \\
        Documentazione   & \textrm{\LaTeX} \\
        Diario di lavoro & \textrm{\LaTeX} \\
        Pianificazione   & MS Excel 2016 \\
        \midrule
        \textrm{\LaTeX} engine & \textrm{\XeLaTeX} \\
        ECAD                   & Altium Designer 2017 \\
        Embedded toolchain     & Microchip XC, MPLabX \\
        Desktop Toolchain      & QtCreator, g++, MinGW \\
        \bottomrule
    \end{tabularx}
\end{table}
}

%\subsection{Hardware}
Per i valori non specificati sono utilizzati i predefiniti del software ECAD.
{\renewcommand\arraystretch{1.2}
\begin{table}[H] \centering
    \caption{Norme di progetto: Hardware}
    \begin{tabularx}{\textwidth}{X r l}
        \toprule
        \bfseries Regola & \bfseries Valore & \bfseries Unit\`a \\
        \midrule
        Number of Layers        &   2 & -- \\
        Silkscreen / Overlay    &  No & -- \\ 
        Minimum trace width     &  30 & mil \\
        Maximum trace width     &  60 & mil \\
        Minimum trace clearance &  20 & mil \\
        Minimum power rail width & 50 & mil \\
        Minimum pad diameter    &  80 & mil \\
        Minimum pad hole diameter & 25 & mil \\
        \bottomrule
    \end{tabularx}
\end{table}
}

%\subsection{Programmazione}
\begin{table}[H] \centering
    \caption{Norme di progetto: Programmazione}
    \begin{tabularx}{\textwidth}{X l}
        \toprule
        \bfseries Regole per programmazione embedded & \\
        \midrule
        Paradigma & Imperativo sequenziale \\
        Convenzione per i nomi & \texttt{snake\_case}, sempre minuscolo \\
        Tabulatore & 4 spazi \\
        Tabulato con gli spazi & S\`i \\
        \midrule
        \bfseries Regole per programmazione desktop & \\
        \midrule
        Paradigma  & Imperativo ad oggetti (OOP) \\
        Convenzione per i nomi & Convenzioni di Qt \\
        Tabulatore & 4 spazi \\
        Tabulato con gli spazi & S\`i \\
        \bottomrule
    \end{tabularx}
\end{table}
