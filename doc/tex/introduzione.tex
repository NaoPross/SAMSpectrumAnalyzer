\chapter{Introduzione}

\section{Contesto}
Per portare a termine il percorso formativo per un attestato di capacit\`a
federale presso la Scuola Arti e Mestieri di Bellinzona \`e richiesto lo
sviluppo individuale di un progetto di produzione di un prodotto.
Per interesse personale nella matematica della trasformata di Fourier mi \`e
stato assegnato di sviluppare un analizzatore spettrale.

\section{Requisiti}
\`E richiesto di sviluppare circuito per analizzare lo spettro dei segnali di
frequenza fino a 10\,kHz. Il dispositivo dovr\`a avere 3 possibili sorgenti:
RCA/Cinch e 2 Audio Jack per un microfono e per una sorgente di audio
generica. \`E inoltre richiesto che il calcolo dei dati dello spettrogramma
sia eseguito da un microcontroller della Microchip, collegato a due
altri dispositivi quali, una display e ad un computer in RS232, per poter
visualizzare lo spettrogramma computato.

\section{Concetto matematico}


\section{Norme di progetto}
