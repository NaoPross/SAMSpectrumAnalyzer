\chapter{Trasformata di Fourier}

\section{Nozioni preliminarie}

\subsection{Regressione lineare con il metodo dei minimi quadrati}
La regressione lineare \`e un'approssimazione di una serie di dati ad una
funzione lineare. Questa retta di approssimazione pu\`o essere calcolata in
molteplici modi, per questo progetto \`e di interesse utilizzare il
\emph{metodo dei minimi quadrati}.  Sar\`a dunque esplicato come trovare i
coefficienti di una retta a \(m+1\) termini partendo da \(N\) punti di
riferimento.
\begin{equation}
    r(x, a_0, \dots, a_m) = a_0 + x\sum_{i=1}^{m}a_i
\end{equation}
Consideriamo di avere gli insiemi \(X\) e \(Y\) entrambi con \(N\) termini di
cui si prende le coppie ordinate di valori \((x_k, y_k)~ x_k\in X,\,y_k\in
Y\), ossia i punti dato di cui eseguire la regressione.  Il metodo dei minimi
quadrati trova i coefficienti della retta minimizzando il quadrato della
differenza tra il valore stimato dalla retta \(r(x_k)\) e il valore reale
\(y_k\).
\begin{equation*}
    \min((r(x_k) - y_k)^2)\quad \forall x_k\in X,\, y_k\in Y
\end{equation*}
Definiamo quindi la funzione da minimizzare \(\varepsilon\)
\begin{equation}
    \varepsilon(a_0, \dots, a_m) = \sum_{k=1}^{N}\Big[r(x_k, a_0, \dots, a_m)  - y_k\Big]^2
\end{equation}
Da cui si computa le derivati parziali rispetto ai coefficienti ricercati,
ottenendo un sistema di equazioni lineare. Ci\`o corrisponde anche ad
affermare che il \emph{gradiente} di \(\varepsilon\) \`e un vettore
\(\in\mathbb{R}^{m+1}\) con tutte le componenti a 0.
\begin{equation*}
    \nabla\varepsilon = \langle 0, \dots, 0 \rangle
\end{equation*}
A questo punto si pu\`o procedere risolvendo il sistema con l'algebra lineare
definendo la matrice di trasformazione \(\mathbf{A}\) e il vettore dei termini
noti \(\vec{u}\)
\begin{equation*}
    \nabla \varepsilon = \mathbf{A}
        \langle a_0, \dots,  a_m \rangle + \vec{u} \iff
    \langle a_0, \dots, a_m \rangle =
        \mathbf{A}^{-1}(-\vec{u})
\end{equation*}

\subsection{Funzione armonica}
Una funzione armonica, sinusoidale, pu\`o essere descritta in molteplici modi.
Iniziamo dunque osservando le forme pi\`u semplici, ossia la forma
trigonometrica.
\begin{align} \label{eq:harmonics-trig}
    f(x) &= a\cdot\sin (\omega x + \varphi) \\
    f(x) &= b\cdot\cos(\omega x + \vartheta)
\end{align}
Conoscendo la formula di Eulero \eqref{eq:euler}
\begin{equation} \label{eq:euler}
    e^{i\varphi} = \cos(\varphi) + i\cdot\sin(\varphi)
\end{equation}
possiamo riscrivere \(f(x)\) nei seguenti modi
\begin{align} \label{eq:harmonics-complex}
    f(x) &= \frac{a}{2i}\cdot(e^{i(x\omega + \varphi)} - e^{-i(x\omega + \varphi)}) \\
    f(x) &= \frac{b}{2}\cdot(e^{i(x\omega + \vartheta)} + e^{-i(x\omega + \vartheta)})
\end{align}

\subsection{Propriet\`a di ortogonalit\`a del seno e del coseno}
Per avere delle fondamenta solide prima dell'introduzione dell'argomento
principale, sar\`a dimostrata l'ortogonalit\`a delle due funzioni
trigonometriche mediante alcune verit\`a matematiche su degli integrali
definiti.  Per tutti i casi seguenti definiamo \(T\) come il periodo della
funzione periodica.

\begin{align*}
    & \int_0^T \sin(\frac{m2\pi x}{T})\,\dd{x} = 0
        \quad \forall m \in \mathbb{Z} \\
    %
    & \int_0^T \cos(\frac{m2\pi x}{T})\,\dd{x} = 0
        \quad \forall m \in \mathbb{Z^*} \\
    %
    & \int_0^T \sin(\frac{m2\pi x}{T})\cos(\frac{n2\pi x}{T})\,\dd{x} = 0
        \quad \forall m,n \in \mathbb{Z} \\
    %
    & \int_0^T \sin(\frac{m2\pi x}{T})\sin(\frac{n2\pi x}{T})\,\dd{x} = 0 
        \quad \forall m,n \in \mathbb{Z}~|~m\neq \pm n \\
    %
    & \int_0^T \sin^2(\frac{m2\pi x}{T})\,\dd{x} = \frac{T}{2}
        \quad \forall m \in \mathbb{Z} \\
    %
    & \int_0^T \cos(\frac{m2\pi x}{T})\cos(\frac{n2\pi x}{T})\,\dd{x} = 0 
        \quad \forall m,n \in \mathbb{Z}~|~m\neq \pm n \\
    %
    & \int_0^T \cos^2(\frac{m2\pi x}{T})\,\dd{x} = \frac{T}{2}
        \quad \forall m \in \mathbb{Z^*} \\
\end{align*}

\subsubsection{Dimostrazioni}


\section{Polinomio Trigonometrico}


\section{Serie di Fourier}
La serie di Fourier, nominata tale in onore a Jean-Baptise Joseph Fourier, di
una funzione \`e descritta nel modo seguente.
\begin{equation} \label{eq:fourier-series}
    f(x) = a_0 + \sum_{n=1}^{\infty}\Big [
        a_n\cdot\cos(\frac{n2\pi x}{T}) +
        b_n\cdot\sin(\frac{n2\pi x}{T}) \Big ]
\end{equation}
Con questa equazione Fourier ha teorizzato che \`e possibile rappresentare
qualsiasi funzione come una combinazione lineare di armoniche di frequenze
multiple di una frequenza di base. Con la seguente identi\`a trigonometrica
\`e possibile anche descrivere la serie con una notazione pi\`u compatta.
\begin{equation*}
    a\cdot\cos(\alpha) + b\cdot\sin(\alpha) = A\cdot\cos(\alpha-\vartheta)
\end{equation*}
Per \(A = \sqrt{a^2+b^2}\), \(\cos(\vartheta)=\frac{b}{A}\) e
\(\sin(\vartheta)=\frac{b}{A}\). Dunque
\begin{align}
    f(x) &= a_0 + \sum_{n=1}^{\infty} 
        A_n\cdot\cos(\frac{n2\pi x}{T} - \vartheta_n) \\
    f(x) &= a_0 + \sum_{n=1}^{\infty} 
        A_n\cdot\sin(\frac{n2\pi x}{T} + \varphi_n)
\end{align}

\section{Trasformata di Fourier discreta}

\section{Trasformata di Fourier}

\section{Fast Fourier Transform}
\subsection{Motivazioni e Complessit\`a temporale}
\subsection{Propriet\`a dei numeri complessi}


