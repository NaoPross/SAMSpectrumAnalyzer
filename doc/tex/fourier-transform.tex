\chapter{Trasformata di Fourier}

\section{Nozioni preliminarie}

\subsection{Regressione lineare con il metodo dei minimi quadrati}
La regressione lineare \`e un'approssimazione di una serie di dati ad una
funzione lineare. Questa retta di approssimazione pu\`o essere calcolata in
molteplici modi, per questo progetto \`e di interesse utilizzare il
\emph{metodo dei minimi quadrati}.  Sar\`a dunque esplicato come trovare i
coefficienti di una retta a \(m+1\) termini interpolando \(N\) punti.
\begin{equation}
    r(x, a_0, \dots, a_m) = a_0 + x\sum_{i=1}^{m}a_i
\end{equation}
Consideriamo di avere gli insiemi \(X\) e \(Y\) entrambi con \(N\) termini di
cui si prende le coppie ordinate di valori, ossia i punti da interpolare.  Il
metodo dei minimi quadrati trova i coefficienti della retta minimizzando il
quadrato della differenza tra il valore stimato dalla retta \(y = r(x_k)\) e il
valore reale \(y_k\).
\begin{equation*}
    \min((r(x_k) - y_k)^2)\quad \forall x_k\in X,\, y_k\in Y
\end{equation*}
Definiamo quindi la funzione da minimizzare \(\varepsilon\)
\begin{equation}
    \varepsilon(a_0, \dots, a_m) = \sum_{k=1}^{N}\Big[r(x_k, a_0, \dots, a_m)  - y_k\Big]^2
\end{equation}
Da cui si computa le derivati parziali rispetto ai coefficienti ricercati,
ottenendo un sistema di equazioni lineare poich\`e si cerca per ogni derivata
quando essa equivale a 0. Ci\`o corrisponde anche ad affermare che il
\emph{gradiente} di \(\varepsilon\) \`e un vettore
\(\in\mathbb{R}^{m+1}\) con tutte le componenti a 0.
\begin{equation*}
    \nabla\varepsilon = (0, \dots, 0 )
\end{equation*}
A questo punto si pu\`o procedere risolvendo il sistema con l'algebra lineare
definendo la matrice di trasformazione \(\mathbf{A}\) e il vettore dei termini
noti \(\vec{u}\)
\begin{equation*}
    \nabla \varepsilon = \mathbf{A}
        ( a_0, \dots, a_m ) + \vec{u} \iff
        ( a_0, \dots, a_m ) =
        \mathbf{A}^{-1}(-\vec{u})
\end{equation*}

\subsection{Funzione armonica}
Una funzione armonica, sinusoidale, pu\`o essere descritta in molteplici modi.
Nella forma trigonometrica, data la pulsazione \(\omega = 2\pi f = \dfrac{2\pi}{T}\)
e la fase \(\varphi = \dfrac{\pi}{2} - \vartheta\).
\begin{align*}
    f(x) &= a\cdot\sin (\omega x + \varphi) \\
    f(x) &= b\cdot\cos(\omega x + \vartheta)
\end{align*}
Conoscendo la formula di Eulero 
\(
    e^{i\varphi} = \cos(\varphi) + i\cdot\sin(\varphi)
\)
possiamo riscrivere \(f(x)\) utilizzando la forma complessa del seno e del coseno.
\begin{align*}
    f(x) &= \frac{a}{2i}\cdot(e^{i(x\omega + \varphi)} - e^{-i(x\omega + \varphi)}) \\
    f(x) &= \frac{b}{2}\cdot(e^{i(x\omega + \vartheta)} + e^{-i(x\omega + \vartheta)})
\end{align*}

\subsection{Propriet\`a di ortogonalit\`a del seno e del coseno}
Per avere delle fondamenta solide prima dell'introduzione dell'argomento
principale, saranno dimostrate le propriet\`a di ortogonalit\`a del seno e
coseno considerando il periodo \(T\) in uno spazio euclideo.

\subsubsection{Intuizione geometrica}

\begin{figure}[H] \centering
    \begin{tikzpicture}
        \begin{axis}[
            width = .5\textwidth,
            height = .5\textwidth,
            axis x line = center,
            axis y line = center,
            xlabel = {\(x\)},
            xlabel style = {right},
            ylabel = {\(y\)},
            ylabel style = {above},
            grid=both,
            grid style = dotted,
            % axis equal,
            xticklabels = \empty,
            extra x ticks = {-10,-9,...,10},
            extra x tick labels= \empty,
            extra y ticks = {-10,-9,...,10},
            extra y tick labels= \empty,
            samples = 200,
            domain=0:2*pi,
            ymin=-1.25, ymax=1.25,
        ]
            \path[name path=axis] (axis cs:0,0) -- (axis cs: 2*pi,0);

            \addplot[red, name path=sin]{sin(deg(x))};
            \addplot[red, opacity=.1] fill between [
                of=sin and axis,
                soft clip={domain=0:2*pi}
            ];

            \addplot[blue, name path=cos]{cos(deg(x))};
            \addplot[blue, opacity=.1] fill between [
                of=cos and axis,
                soft clip={domain=0:2*pi}
            ];
        \end{axis}
    \end{tikzpicture}
    \begin{tikzpicture}[>=latex]
        \begin{axis}[
            width = .5\textwidth,
            height = .5\textwidth,
            axis x line = center,
            axis y line = center,
            xlabel = {\(\cos(t)\)},
            xlabel style = {right},
            ylabel = {\(\sin(t)\)},
            ylabel style = {above},
            grid=both,
            grid style = dotted,
            axis equal,
            xticklabels = \empty,
            yticklabels = \empty,
            extra x ticks = {-10,-9,...,10},
            extra x tick labels= \empty,
            extra y ticks = {-10,-9,...,10},
            extra y tick labels= \empty,
            xmin=-1.25, xmax=1.25,
            ymin=-1.25, ymax=1.25,
            samples=800,
        ]
            \addplot[mark=none,domain=0:360] ({cos(x)}, {sin(x)});

            \def\angle{50}
            \pgfmathsetmacro\ax{cos(\angle)}
            \pgfmathsetmacro\ay{sin(\angle)}

            \coordinate (a) at (\ax,\ay);
            \coordinate (o) at (0,0);
            \coordinate (z) at (1,0);

            \pic[draw, ->, "\(t\)", angle eccentricity=1.5] {
                angle = z--o--a
            };

            \draw[thick, violet] (0,0) -- (a);
            \draw[thick, red] (a) -- (a |- o);
            \draw[thick, blue] (o) -- (a |- o);

        \end{axis}

    \end{tikzpicture}

    \caption{Funzioni \(\sin\) e \(\cos\) \label{fig:sin-cos-orth-plot}}
\end{figure}

Si pu\`o osservare intuitivamente dal cerchio unitario nella figura
\ref{fig:sin-cos-orth-plot} che le funzioni seno e coseno sono ortogonali tra
loro. Dal grafico a sinistra possiamo inoltre osservare che l'area (integrale)
di un periodo \`e sempre nulla.

\subsubsection{Dimostrzioni algebriche}
\begin{enumerate}
\item {
    Area di \(\sin\) in un periodo.
    \begin{align*}
        \int_0^T \sin(\frac{m2\pi x}{T})\,\dd{x} &= 0
        \quad \forall m \in \mathbb{Z} \\
        \\
        %
        \int_0^T \sin(\frac{m2\pi x}{T})\,\dd{x}
        &= \bigg [-\frac{T}{2\pi m }\cdot\cos\big(\frac{2\pi}{T}mx\big)\bigg]^T_0 \\
        &= -\frac{T}{2\pi m}\cdot\cos\big(2\pi m\big) 
            +\frac{T}{2\pi m}\cdot\cos\big(0\big) \\
        &= 0
    \end{align*}
}

\item {
    Area di \(\cos\) in un periodo.
    \begin{align*}
        \int_0^T \cos(\frac{m2\pi x}{T})\,\dd{x} &= 0
        \quad \forall m \in \mathbb{Z}^* \\
        \\
        %
        \int_0^T \cos(\frac{m2\pi x}{T})\,\dd{x}
        &= \bigg [\frac{T}{2\pi m}\cdot\sin\big(\frac{2\pi}{T}mx\big)\bigg]^T_0 \\
        &= \frac{T}{2\pi m}\cdot\sin\big(2\pi m\big) 
            +\frac{T}{2\pi m}\cdot\sin\big(0\big) \\
        &= \begin{cases}
            0 & \iff m \neq 0 \\
            T & \iff m = 0 
        \end{cases}
    \end{align*}
}

\item {
    Prodotto tra \(\sin\) e \(\cos\).
    \begin{align*}
        \int_0^T \sin(\frac{m2\pi x}{T})\cos(\frac{n2\pi x}{T})\,\dd{x} &= 0
        \quad \forall m,n \in \mathbb{Z} \\
        \\
        %
        \int_0^T \sin(\frac{m2\pi x}{T})\cos(\frac{n2\pi x}{T})\,\dd{x} &=
        \underbrace{\frac{1}{2}\int_0^T\sin\bigg[\frac{2\pi}{T}(n+m)x\bigg]\,\dd{x}}_{0} - 
        \underbrace{\frac{1}{2}\int_0^T\sin\bigg[\frac{2\pi}{T}(n-m)x\bigg]\,\dd{x}}_{0} \\
        %
        &= 0 \\
    \end{align*}
}

\item {
    Prodotto tra due \(\sin\) di frequenze diverse.
    \begin{align*}
        \int_0^T \sin(\frac{m2\pi x}{T})\sin(\frac{n2\pi x}{T})\,\dd{x} &= 0 
        \quad \forall m,n \in \mathbb{Z}~|~m\neq \pm n \\
        \\
        %
        \int_0^T \sin(\frac{m2\pi x}{T})\sin(\frac{n2\pi x}{T})\,\dd{x} &= 
        \underbrace{\frac{1}{2}\int_0^T\cos\bigg[\frac{2\pi}{T}(n-m)x\bigg]\,\dd{x}}_{m-n~\neq~0\implies 0} -
        \underbrace{\frac{1}{2}\int_0^T\cos\bigg[\frac{2\pi}{T}(n+m)x\bigg]\,\dd{x}}_{m+n~\neq~0\implies 0} \\
        %
        &= \begin{cases}
            0 & \iff n \neq \pm m \\
            T/2 & \iff n = m \\
            - T/2 & \iff n = - m \\
        \end{cases}
    \end{align*}
}

\item {
    Prodotto tra due \(\cos\) di frequenze diverse.
    \begin{align*}
        \int_0^T \cos(\frac{m2\pi x}{T})\cos(\frac{n2\pi x}{T})\,\dd{x} &= 0 
        \quad \forall m,n \in \mathbb{Z}^*~|~m\neq \pm n \\
        \\
        %
        \int_0^T \cos(\frac{m2\pi x}{T})\cos(\frac{n2\pi x}{T})\,\dd{x} &= 
        \underbrace{\frac{1}{2}\int_0^T\cos\bigg[\frac{2\pi}{T}(n+m)x\bigg]\,\dd{x}}_{m+n~\neq~0\implies 0} +
        \underbrace{\frac{1}{2}\int_0^T\cos\bigg[\frac{2\pi}{T}(n-m)x\bigg]\,\dd{x}}_{m-n~\neq~0\implies 0} \\
        %
        &= \begin{cases}
            0 & \iff n \neq \pm m \\
            T/2 & \iff n = \pm m \\
        \end{cases}
    \end{align*}
}

\item {
    \(\sin\) e \(\cos\) raggruppate nella forma complessa con la formula di Eulero.
    \begin{align*}
        \int_0^T e^{i2\pi kx/T}\,\dd{x} &= 0
        \quad \forall k \in \mathbb{Z}^* \\
        \\
        %
        \int_0^T e^{i2\pi kx/T}\,\dd{x} &=
        \underbrace{\int_0^T\cos(\frac{2\pi}{T} kx)\,\dd{x}}_{k\neq 0 \implies 0} + 
        \underbrace{i\int_0^T\sin(\frac{2\pi}{T} kx)\,\dd{x}}_{0}  \\
        %
        &= \begin{cases}
            0 & \iff k \neq 0 \\
            T & \iff k = 0
        \end{cases}
    \end{align*}
}
\end{enumerate}

\section{Polinomio Trigonometrico}
\subsection{Polinomio Trigonometrico}
Analogamente a come \`e definito un polinomio \(P\) ``normale'', una somma di
termini dal grado 0 fino al grado \(N\), \`e possibile definire anche un
polinomio trigonometrico \(T\).
\[
    P_N(x) = \sum_{n=0}^N a_n x^n \qquad a_n \in \mathbb{R},~ a_N \neq 0
\]
\[
    T_N(x) = \sum_{n=0}^N c_n e^{i\omega nx} 
        \qquad c_n \in\mathbb{C},~\omega\in\mathbb{R}, ~ c_N \neq 0
\]
Questo polinomio \`e detto \emph{trigonometrico} perch\'e utilizzando la
formula di Eulero \(e^{i\varphi} = \cos(\varphi) + i\sin(\varphi)\) si pu\`o
espandere nel seguente modo.
\[
    T_N(x) = \sum_{n=0}^N\big [a_n\cdot\cos(\omega nx) + ib_n\cdot\sin(\omega nx)]
    \qquad a_n, b_n \in \mathbb{C}
\]

\subsection{Polinomio Trigonometrico Reale} \label{sec:real-trig}
\`E definito inoltre il polinomio trogonometrico \emph{reale} come
\[
    T_N(x) = \sum_{n=0}^N\big [a_n\cdot\cos(\omega nx) + b_n\cdot\sin(\omega nx)]
    \qquad a_n, b_n \in \mathbb{R}
\]
Mediante delle identit\`a trigonometriche pu\`o essere riscritto anche nel
modo seguente.
\[
    T_N(x) = \sum_{n=0}^N A_n\cdot\cos(\omega nx - \varphi)
\]
Oppure nella sua forma complessa in cui \(c_k = \dfrac{a_k}{2} + \dfrac{b_k}{2i}\).
\[
    T_N(x) = \sum_{n=0}^N c_n\,e^{i\omega nx} + \bar{c_n}\,e^{-i\omega nx}
    \qquad c_n \in \mathbb{C}
\]
Definendo \(c_{-n} = \bar{c_n}\) \`e possibile ridurre la notazione.
\[
    T_N(x) = \sum_{n=-N}^N c_n\,e^{i\omega nx}
\]

In tutti i casi possiamo osservare che il polinomio trogonometrico \`e una
somma di sinusoidi di frequenze multiple ad una base \(f = \omega/2\pi\).  Se
descritto mediante la terminologia dell'algebra lineare, si pu\`o anche
osservare che un polinomio trigonometrico \`e una combinazione lineare nello
spazio funzionale dalle basi ortonormate \(\sin(\omega x)\) e \(\cos(\omega
x)\).


\section{Serie di Fourier}
Un polinomio trigonometrico reale di infiniti termini \`e una serie di Fourier,
in onore al matematico francese  J. B. Fourier.
\[
	S(x) = \sum_{n=-\infty}^\infty c_n\,e^{i\omega nx} 
         = \sum_{n=0}^\infty a_n\cdot\cos(\omega nx) + b_n\cdot\sin(\omega nx)
\]

\subsection{Convergenza della serie}
La convergenza puntuale della serie di Fourier \`e dimostrabile dalle
condizioni del teorema di Dirichlet. Purtroppo questa dimostrazione richiede
una grande quantit\`a di nozionistica matematica che non pu\`o essere riassunta
in pochi paragrafi.  In allegato e nelle referenze sono presenti dei documenti
che analizzano e dimostrano questa propriet\`a.

% \[
% 	S(x) = \sum_{n=-\infty}^\infty c_n\,e^{i\omega nx} 
% 		 = \frac{1}{2}\big[\lim_{y\to x+}f(y) + \lim_{y\to x-}f(y)\big]
% 		 \qquad \forall x
% \]

Da questo punto \`e da dare per assunto che \`e dimostrata la convergenza
puntuale della serie di Fourier di una funzione reale a condizione che:
\begin{enumerate}
	\item \(|f(x)|\) (valore assoluto) \`e integrabile in un periodo.
	\item La funzione deve essere di variazione limitata, ossia devono esserci
		un numero finito di minimi e massimi in un qualsiasi intervallo.
	\item La funzione deve avere un numero finito di discontinuti\`a in un
		qualsiasi intervallo chiuso e le discontinuit\`a non devono essere infinite.
\end{enumerate}

\section{Trasformata di Fourier discreta}
\begin{figure}[H] \centering
    \includegraphics[]{figures/fourier-3d-plot.tikz}
	\caption{
		Rappresentazione grafica della transformata di Fourier \cite{ftplot}
		\label{fig:dftplot}
	}
\end{figure}
La Trasformata di Fouerier Discreta (DFT) \`e l'operazione matematica che
permette di trovare i coefficienti della serie di Fourier o di un polinomio
trigonometrico, per approssimare al meglio una funzione. Se presa a sestante
per\`o essa essendo una \emph{trasformata} pu\`o essere osservata anche come
una funzione che da uno spettro \emph{discreto} di una funzione. La densit\`a
spettrale ottenuta dalla DFT \`e dipende dalla frequenza di base scelta e nel
caso di un polinomio trigonometrico, anche dal numero di termini di
quest'ultimo.

\noindent\fbox{\begin{minipage}{\textwidth}\[
    X_k = c_k \cdot T = \int_0^T f(x)\cdot e^{-i\omega kx}\,\dd{x}
\]\end{minipage}}

Nella figura \ref{fig:dftplot}, possiamo osservare sul grafico \(x\bot z\) in
rosso una funzione, ed il suo spettro sull'asse \(y\bot z\) in blu. Si pu\`o
intuire che pi\`u

\subsection{Derivazione della DFT}
Per trovare la DFT, supponiamo di voler approssimare una funzione,
utilizzando il metodo dei minimi quadrati, con un polinomio trigonometrico
reale.
\[
    T_N(x) =  \sum_{n=0}^N A_n\cdot\cos(\omega nx -\varphi) \qquad \omega=2\pi f
\]
Sappiamo che questo pu\`o essere scritto anche nel seguente modo.
\[
    T_N(x) =  \sum_{n=0}^N a_n\cdot\cos(\omega nx)  + b_n\cdot\sin(\omega nx)
\]

Dunque per trovare i termini \(a_0,\dots,a_N\) e \(b_0,\dots,b_N\) definiamo
una funzione \(\varepsilon\) da minimizzare con il metodo dei minimi quadrati.
\[
    \varepsilon = \int_0^T\big[T_N(x) - f(x)\big]^2\,\dd{x}
\]
Per generalizzare sar\`a dimostrato come trovare il \(k\)-esimo termine.

\subsubsection{Termini dei coseni}
I termini dei coseni \(a_k\) sono ottenuti eguagliando la derivata parziale di
\(\varepsilon\) a zero.
\begin{align*}
    \frac{\partial\varepsilon}{\partial a_k} = 0 &= 
    \frac{\partial}{\partial a_k} \int_0^T\big[T_N(x) - f(x)\big]^2\,\dd{x}
    \\
    0 &= \int_0^T \frac{\partial}{\partial a_k} \bigg[
            \sum_{n=0}^N a_n\cos(\omega nx) +  b_n\sin(\omega nx) - f(x)
        \bigg]^2\,\dd{x}
    \\
    0 &= \int_0^T 2\bigg[
            \sum_{n=0}^N a_n\cos(\omega nx) +  b_n\sin(\omega nx) - f(x)
        \bigg]\cdot a_k\cos(\omega kx)\,\dd{x}
    \\
    0 &=
        \underbrace{\int_0^T \cos(\omega kx)\sum_{n=0}^N a_n\cos(\omega nx) \,\dd{x}}_{k \neq n \implies 0} + 
        \underbrace{\int_0^T \cos(\omega kx)\sum_{n=0}^N b_n\sin(\omega nx) \,\dd{x}}_{0} - 
        \int_0^T \cos(\omega kx)\cdot f(x) \,\dd{x}
    \\
    0 &= a_k\cdot\frac{T}{2} - \int_0^T\cos(\omega kx)\cdot f(x)\,\dd{x}
    \\
    \\
    \frac{a_k}{2} &= \frac{1}{T}\int_0^T\cos(\omega kx)\cdot f(x)\,\dd{x}
\end{align*}

\subsubsection{Termini dei seni}
I termini dei seni \(b_k\) sono ottenuti eguagliando la derivata parziale di
\(\varepsilon\) a zero.
\begin{align*}
    \frac{\partial\varepsilon}{\partial b_k} = 0 &= 
    \frac{\partial}{\partial b_k} \int_0^T\big[T_N(x) - f(x)\big]^2\,\dd{x}
    \\
    0 &= \int_0^T \frac{\partial}{\partial b_k} \bigg[
            \sum_{n=0}^N a_n\cos(\omega nx) +  b_n\sin(\omega nx) - f(x)
        \bigg]^2\,\dd{x}
    \\
    0 &= \int_0^T 2\bigg[
            \sum_{n=0}^N a_n\cos(\omega nx) +  b_n\sin(\omega nx) - f(x)
        \bigg]\cdot b_k\sin(\omega kx)\,\dd{x}
    \\
    0 &=
        \underbrace{\int_0^T \sin(\omega kx)\sum_{n=0}^N a_n\cos(\omega nx) \,\dd{x}}_{0} + 
        \underbrace{\int_0^T \sin(\omega kx)\sum_{n=0}^N b_n\sin(\omega nx) \,\dd{x}}_{k \neq n \implies 0} - 
        \int_0^T \sin(\omega kx)\cdot f(x) \,\dd{x}
    \\
    0 &= b_k\cdot\frac{T}{2} - \int_0^T\sin(\omega kx)\cdot f(x)\,\dd{x}
    \\
    \\
    \frac{b_k}{2} &= \frac{1}{T}\int_0^T\sin(\omega kx)\cdot f(x)\,\dd{x}
\end{align*}

\subsubsection{Termine complesso}
A questo punto \`e possibile raggurppare i termini \(a_k\) e \(b_k\) in un
unico valore complesso \(c_k\), come descritto nella sezione
\ref{sec:real-trig}.
\begin{align*}
    c_k &= \frac{a_k}{2} + \frac{b_k}{2i}
    \\
    c_k &= 
    \frac{1}{T}\int_0^T\cos(\omega kx)\cdot f(x)\,\dd{x} + 
    \frac{1}{Ti}\int_0^T\sin(\omega kx)\cdot f(x)\,\dd{x}
    \\
    c_k &= \frac{1}{T}\int_0^T f(x)\cdot\big[\cos(\omega kx) -i\sin(\omega kx)\big]\,\dd{x}
    \\
    \\
    c_k &= \frac{1}{T}\int_0^T f(x)\cdot e^{-i\omega kx}\,\dd{x}
\end{align*}

\subsection{DFT da un insieme di campioni discreti}
L'operazione della DFT pu\`o essere applicata anche ad un insieme di \(N\) valori
discreti, che per esempio potrebbero originare da un campionamento di un
segnale elettrico. In tal caso la trasformata discreta di Fourier \`e la
seguente.

\noindent\fbox{\begin{minipage}{\textwidth}\[
    X_k = c_k \cdot T = \sum_{n=0}^{N-1} x_n\cdot e^{-i\omega kn}
\]\end{minipage}}

Utilizzando questa notazione con gli indici \`e possibile anche notare che la
TFD pu\`o essere espressa come una matrice \(\mathbf{V}\). Definendo il vettore
\(\vec{x} = (x_0,x_1,\dots,x_{N-1})\) e \(\vec{X} = (X_0,\dots,X_{N-1})\) si ottiene

\noindent\fbox{\begin{minipage}{\textwidth}\[
    \vec{X} = \mathbf{V}\cdot\vec{x}
    \vspace{2mm}
\]\end{minipage}}

\noindent Dove 
\[
    [\mathbf{V}]_{kn} = (e^{i\omega})^{-kn}
\]
ossia, se rappresentata nel suo intero
\[
    \mathbf{V} = 
    \begin{pmatrix}
    1      & 1                      & 1                       & \dots  & 1                        \\[1em]
    1      & (e^{i\omega})^{-1}     & (e^{i\omega})^{-2}      & \dots  & (e^{i\omega})^{-(N-1)}   \\[1em]
    1      & (e^{i\omega})^{-2}     & (e^{i\omega})^{-4}      & \dots  & (e^{i\omega})^{-2(N-1)}  \\[1em]
    \vdots & \vdots                 &  \vdots                 & \ddots & \vdots                   \\[1em]
    1      & (e^{i\omega})^{-(N-1)} & (e^{i\omega})^{-2(N-1)} & \dots  & (e^{i\omega})^{-(N-1)^2} \\[1em]
    \end{pmatrix}
\]

Osservando la matrice si nota subito una simmetria diagonale, che \`e data
dalla  progressione geometrica \([\mathbf{V}]_{ij} = [\mathbf{V}]_i^{j-1}\).
Questa matrice \`e detta una matrice di Vandermonde, da cui \`e stato scelto
il nome \(\mathbf{V}\).

\section{Trasformata di Fourier}
Dalla DFT si \`e sviluppato lo strumento matematico per ottenere lo spettro
discreto di una funzione. Il risultato della DFT per\`o \`e una funzione
discontinua con valori spettrali unicamente multipli della base.

La Trasformata di Fourier estende ulteriormente producendo una funzione di
spettro \emph{continua}.  Un modo intuitivo per ottenere questo requisito, \`e
di far tendere il limite della densit\`a spettrale, data dal periodo, verso
l'infinito.  Per i seguenti passaggi sar\`a utilizzata la funzione \(s\) per
indicare la funzione reale di cui si osserva lo spettro. Partendo
dall'approssimazione data dalla DFT possiamo asserire che
\[
    s(x) \approx \sum_{n=-\infty}^\infty \frac{X_n}{T}
        \cdot\,e^{i2\pi xn/T}
\]
\[
    s(x) \approx \sum_{n=-\infty}^\infty
        \frac{1}{T}\,
        \underbrace{
            \int_{-T/2}^{T/2} s(x)\cdot e^{-i2\pi xn/T}\,\dd{x}
        }_{X_n}
        \cdot\,e^{i2\pi xn/T}
\]


\[
    s(x) = \lim_{T\to\infty} \sum_{n=-\infty}^\infty
        \frac{1}{T}\int_{-T/2}^{T/2} s(x)\cdot e^{-i2\pi xn/T}\,\dd{x}
        \cdot\,e^{i2\pi xn/T}
\]
Osserviamo che
\begin{align*}
    \lim_{T\to\infty} \sum_{n=-\infty}^\infty \frac{1}{T} &= \int_{-\infty}^\infty \dd{f} &
    \lim_{T\to\infty} \frac{n}{T} &= f \quad \text{(variabile continua)}
\end{align*}
Dunque otteniamo
\[
    s(x) = \int_{-\infty}^\infty \dd{f} 
           \underbrace{
               \int_{-\infty}^\infty s(x)\cdot e^{-i2\pi fx}\,\dd{x}
           }_{\mathcal{F}\{s\}}
           \cdot\,e^{i2\pi fx}
\]
Si nota a questo punto che il termine centrale \`e la trasformata di Fourier
ed \`e una funzione dalla variabile continua \(f\) (frequenza).
Si osservi inoltre che sono stati cambiati i limiti di integrazione del
periodo.  Definendo entrambi i limiti in funzione di \(T\) e facendo tendere
\(T\) all'infinito, si ottiene come conseguenza secondaria che la trasformata
non \`e pi\`u limitata ad una funzione periodica.

Formalmente dunque la Trasformata di Fourier, tipicamente notata con
\(\mathcal{F}\) oppure con l'accento circonflesso \(\hat{f}\) sul nome della
funzione, \`e definita come segue.

\noindent\fbox{\begin{minipage}{\textwidth}\[
    \mathcal{F}\{f\} = \hat{f}\,(\omega) 
    = \int_{-\infty}^\infty f(x)\cdot e^{-i\omega x}\,\dd{x}
\]\end{minipage}}

% \section{Interpretazione geometrica}
% \subsection{Spazi funzionali}
% \subsection{Prodotto interno}
% \subsection{Metodo dei minimi quadrati}
% 
% \section{Fast Fourier Transform}
% \subsection{Motivazioni e Complessit\`a temporale}
% \subsection{Propriet\`a dei numeri complessi}

% \section{}
