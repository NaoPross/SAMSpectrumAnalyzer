\chapter{Trasformata di Fourier}

\section{Nozioni preliminarie}

\subsection{Regressione lineare con il metodo dei minimi quadrati}
La regressione lineare \`e un'approssimazione di una serie di dati ad una
funzione lineare. Questa retta di approssimazione pu\`o essere calcolata in
molteplici modi, per questo progetto \`e di interesse utilizzare il
\emph{metodo dei minimi quadrati}.  Sar\`a dunque esplicato come trovare i
coefficienti di una retta a \(m+1\) termini interpolando \(N\) punti.
\begin{equation}
    r(x, a_0, \dots, a_m) = a_0 + x\sum_{i=1}^{m}a_i
\end{equation}
Consideriamo di avere gli insiemi \(X\) e \(Y\) entrambi con \(N\) termini di
cui si prende le coppie ordinate di valori, ossia i punti da interopolare.  Il
metodo dei minimi quadrati trova i coefficienti della retta minimizzando il
quadrato della differenza tra il valore stimato dalla retta \(y = r(x_k)\) e il
valore reale
\(y_k\).
\begin{equation*}
    \min((r(x_k) - y_k)^2)\quad \forall x_k\in X,\, y_k\in Y
\end{equation*}
Definiamo quindi la funzione da minimizzare \(\varepsilon\)
\begin{equation}
    \varepsilon(a_0, \dots, a_m) = \sum_{k=1}^{N}\Big[r(x_k, a_0, \dots, a_m)  - y_k\Big]^2
\end{equation}
Da cui si computa le derivati parziali rispetto ai coefficienti ricercati,
ottenendo un sistema di equazioni lineare. Ci\`o corrisponde anche ad
affermare che il \emph{gradiente} di \(\varepsilon\) \`e un vettore
\(\in\mathbb{R}^{m+1}\) con tutte le componenti a 0.
\begin{equation*}
    \nabla\varepsilon = \langle 0, \dots, 0 \rangle
\end{equation*}
A questo punto si pu\`o procedere risolvendo il sistema con l'algebra lineare
definendo la matrice di trasformazione \(\mathbf{A}\) e il vettore dei termini
noti \(\vec{u}\)
\begin{equation*}
    \nabla \varepsilon = \mathbf{A}
        \langle a_0, \dots,  a_m \rangle + \vec{u} \iff
    \langle a_0, \dots, a_m \rangle =
        \mathbf{A}^{-1}(-\vec{u})
\end{equation*}

\subsection{Funzione armonica}
Una funzione armonica, sinusoidale, pu\`o essere descritta in molteplici modi.
Iniziamo dunque osservando le forme pi\`u semplici, ossia la forma
trigonometrica.
\begin{align*}
    f(x) &= a\cdot\sin (\omega x + \varphi) \\
    f(x) &= b\cdot\cos(\omega x + \vartheta)
\end{align*}
Conoscendo la formula di Eulero 
\[
    e^{i\varphi} = \cos(\varphi) + i\cdot\sin(\varphi)
\]
possiamo riscrivere \(f(x)\) nei seguenti modi
\begin{align*}
    f(x) &= \frac{a}{2i}\cdot(e^{i(x\omega + \varphi)} - e^{-i(x\omega + \varphi)}) \\
    f(x) &= \frac{b}{2}\cdot(e^{i(x\omega + \vartheta)} + e^{-i(x\omega + \vartheta)})
\end{align*}

\subsection{Propriet\`a di ortogonalit\`a del seno e del coseno}
Per avere delle fondamenta solide prima dell'introduzione dell'argomento
principale, saranno dimostrate le propriet\`a di ortogonalit\`a del seno e
coseno. Considerando il periodo \(T\), dunque di frequenza \(2\pi /\,T\).

\paragraph{Intuizione geometrica}



\paragraph{Dimostrzioni algebriche }
\begin{enumerate}
\item {
    \begin{align*}
        \int_0^T \sin(\frac{m2\pi x}{T})\,\dd{x} &= 0
        \quad \forall m \in \mathbb{Z} \\
        %
        \int_0^T \sin(\frac{m2\pi x}{T})\,\dd{x}
        &= \bigg [-\frac{T}{2\pi m }\cdot\cos\big(\frac{2\pi}{T}mx\big)\bigg]^T_0 \\
        &= -\frac{T}{2\pi m}\cdot\cos\big(2\pi m\big) 
            +\frac{T}{2\pi m}\cdot\cos\big(0\big) \\
        &= 0
    \end{align*}
}

\item {
    \begin{align*}
        \int_0^T \cos(\frac{m2\pi x}{T})\,\dd{x} &= 0
        \quad \forall m \in \mathbb{Z}^* \\
        %
        \int_0^T \cos(\frac{m2\pi x}{T})\,\dd{x}
        &= \bigg [\frac{T}{2\pi m}\cdot\sin\big(\frac{2\pi}{T}mx\big)\bigg]^T_0 \\
        &= \frac{T}{2\pi m}\cdot\sin\big(2\pi m\big) 
            +\frac{T}{2\pi m}\cdot\sin\big(0\big) \\
        &= 0
    \end{align*}
    Nota: Se \(m = 0\) 
    \[\int_0^T \cos(\frac{m2\pi x}{T})\,\dd{x} = T\]
}

\item {
}
\end{enumerate}

% \begin{align*}
%     & \int_0^T \sin(\frac{m2\pi x}{T})\,\dd{x} = 0
%         \quad \forall m \in \mathbb{Z} \\
%     %
%     & \int_0^T \cos(\frac{m2\pi x}{T})\,\dd{x} = 0
%         \quad \forall m \in \mathbb{Z^*} \\
%     %
%     & \int_0^T \sin(\frac{m2\pi x}{T})\cos(\frac{n2\pi x}{T})\,\dd{x} = 0
%         \quad \forall m,n \in \mathbb{Z} \\
%     %
%     & \int_0^T \sin(\frac{m2\pi x}{T})\sin(\frac{n2\pi x}{T})\,\dd{x} = 0 
%         \quad \forall m,n \in \mathbb{Z}~|~m\neq \pm n \\
%     %
%     & \int_0^T \sin^2(\frac{m2\pi x}{T})\,\dd{x} = \frac{T}{2}
%         \quad \forall m \in \mathbb{Z} \\
%     %
%     & \int_0^T \cos(\frac{m2\pi x}{T})\cos(\frac{n2\pi x}{T})\,\dd{x} = 0 
%         \quad \forall m,n \in \mathbb{Z}~|~m\neq \pm n \\
%     %
%     & \int_0^T \cos^2(\frac{m2\pi x}{T})\,\dd{x} = \frac{T}{2}
%         \quad \forall m \in \mathbb{Z^*} \\
% \end{align*}

\section{Polinomio Trigonometrico}
Analogamente a come \`e definito un polinomio \(P\) ``normale'', una somma di
termini dal grado 0 fino al grado \(N\), \`e possibile definire anche un
polinomio trigonometrico \(T\).
\[
    P_N(x) = \sum_{n=0}^N a_n x^n \qquad a_n \in \mathbb{R},~ a_N \neq 0
\]
\[
    T_N(x) = \sum_{n=0}^N c_n e^{i\omega nx} 
        \qquad c_n \in\mathbb{C},~\omega\in\mathbb{R}, ~ c_N \neq 0
\]
Questo polinomio \`e detto \emph{trigonometrico} perch\`e utilizzando la
formula di eulero \(e^{i\varphi} = \cos(\varphi) + i\sin(\varphi)\) si pu\`o
espandere nel seguente modo.
\[
    T_N(x) = \sum_{n=0}^N\big [a_n\cdot\cos(\omega nx) + ib_n\cdot\sin(\omega nx)]
    \qquad a_n, b_n \in \mathbb{C}
\]
\`E definito inoltre il polinomio trogonometrico \emph{reale} come
\[
    T_N(x) = \sum_{n=0}^N\big [a_n\cdot\cos(\omega nx) + b_n\cdot\sin(\omega nx)]
    \qquad a_n, b_n \in \mathbb{R}
\]
Quest'ultimo mediante delle identit\`a trigonometriche pu\`o essere riscritto
anche nel modo seguente.
\[
    T_N(x) = \sum_{n=0}^N A_n\cdot\cos(\omega nx - \varphi)
\]

In tutti i casi possiamo osservare che il polinomio trogonometrico \`e una
somma di sinusoidi di frequenze multiple ad una base \(f = \omega/2\pi\).
Se descritto mediante la terminologia dell'algebra lineare, si pu\`o anche
osservare che un polinomio trigonometrico \`e una combinazione lineare nello
spazio funzionale ortonormato dalle basi \(\sin(\omega nx)\) e \(\cos(\omega
nx)\).


\section{Serie di Fourier}

\section{Trasformata di Fourier discreta}
La Trasformata di Fouerier Discreta (DFT) \`e l'operazione matematica che
permette di trovare i coefficienti della serie di Fourier o di un polinomio
trigonometrico, per approssimare al meglio una funzione. Se presa a sestante
per\`o essa essendo una \emph{trasformata} pu\`o essere osservata anche come
una funzione che da uno spettro \emph{discreto} di una funzione. La densit\`a
spettrale ottenuta dalla DFT \`e dipende dalla frequenza di base scelta e nel
caso di un polinomio trigonometrico, anche dal numero di termini di
quest'ultimo.
\[
    \hat{f_k} = \frac{1}{T}\int_0^T f(x)\cdot e^{-ik2\pi fx}\,\dd{x}
\]

\subsection{Derivazione della DFT}
Per trovare la DFT, supponiamo di voler approssimare una funzione \(f\),
utilizzando il metodo dei minimi quadrati, con un polinomio trigonometrico
reale.
\[
    T_N(x) =  \sum_{n=0}^N A_n\cdot\cos(\omega nx -\varphi) \qquad \omega=2\pi f
\]
Sappiamo che questo pu\`o essere scritto anche nel seguente modo.
\[
    T_N(x) =  \sum_{n=0}^N a_n\cdot\cos(\omega nx)  + b_n\cdot\sin(\omega nx)
\]

Dunque per trovare i termini \(a_0,\dots,a_N\) e \(b_0,\dots,b_N\) definiamo
una funzione \(\varepsilon\) da minimizzare con il metodo dei minimi quadrati.
\[
    \varepsilon = \int_0^T\big[T_N(x) - f(x)\big]^2\,\dd{x}
\]

Per generalizzare sar\`a dimostrato come trovare il \(k\)-esimo termine.
\[
    \frac{\partial\varepsilon}{\partial a_k} =
    \frac{\partial}{\partial a_k} \int_0^T\big[T_N(x) - f(x)\big]^2\,\dd{x}
    = 0
\]
\[
    \frac{\partial\varepsilon}{\partial b_k} =
    \frac{\partial}{\partial b_k} \int_0^T\big[T_N(x) - f(x)\big]^2\,\dd{x}
    = 0
\]

\subsubsection{Termini dei coseni}
\begin{align*}
    \frac{\partial\varepsilon}{\partial a_k} =
    \frac{\partial}{\partial a_k} \int_0^T\big[T_N(x) - f(x)\big]^2\,\dd{x}
    &= 0 \\
\end{align*}
\subsubsection{Termini dei seni}

\section{Trasformata di Fourier}

\section{Fast Fourier Transform}
\subsection{Motivazioni e Complessit\`a temporale}
\subsection{Propriet\`a dei numeri complessi}


