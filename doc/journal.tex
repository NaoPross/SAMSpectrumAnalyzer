\documentclass[a4paper]{article}

% metadata
% ------------------------------------------------------------------------------

\newcommand{\name}{Naoki Pross}
\newcommand{\instructor}{Rinaldo Geiler, Daniele Kamm}
\newcommand{\project}{Spectrum Analyzer}
\newcommand{\projstart}{12.04.2018}
\newcommand{\projend}{15.04.2018}

% auto
\newcommand{\projperiod}{\projstart{} -- \projend{}}

% preable
% ------------------------------------------------------------------------------

% tables
\usepackage{tabularx}
\usepackage{booktabs}
\usepackage{array}
\usepackage{multirow}
\usepackage{ltablex}
\keepXColumns

% colors
\usepackage[table]{xcolor}
\usepackage{graphicx}

% font
% \usepackage[sfdefault]{noto}
% \usepackage{lmodern}
\renewcommand{\familydefault}{\sfdefault}

% layout
\usepackage[left=2cm, right=2cm, top=3cm, bottom=3cm]{geometry}
\usepackage{fancyhdr}

% header and footer
\pagestyle{fancy}
\fancyhead[L]{CAM-SAM}
\fancyhead[C]{Elettronico}
\fancyhead[R]{22.02.2018}
\fancyfoot[L]{\jobname.tex}
\fancyfoot[C]{\name}
\fancyfoot[R]{\thepage}

\renewcommand\arraystretch{1.5}
\renewcommand\tabcolsep{5pt}
\setlength{\parindent}{0pt}
\setlength{\parskip}{1em}

% macros
\newcommand{\journalentry}[5]{%
    #1 & #2 & #3 & #4 & #5 \\\hline
}

% document
% ------------------------------------------------------------------------------

\begin{document}
    \begin{center}
        \bf \LARGE DIARIO GIORNALIERO
    \end{center}
    
    \vspace{5mm}

    \begin{tabularx}{\textwidth}{| lX | lX |}
        \hline
        \bfseries Candidato: & \name &
        \bfseries Progetto: & \project
        \\\hline
        \bfseries Formatore: & \instructor &
        \bfseries Periodo: & \projperiod
        \\\hline
    \end{tabularx}
   
    \vspace{5mm}
    
    \begin{tabularx}{\textwidth}{| c | c | c | p{.4\textwidth} | X |}
        \hline    
        \rowcolor{gray!30}
        \bfseries Giorno &
        \bfseries Data & 
        \bfseries Ore &
        \bfseries Descrizione attivit\`a &
        \bfseries Osservazioni
        \\
        \rowcolor{gray!30}
        & & &
        (Attivit\`a eseguite, metodi adottati, decisioni prese,
        dimostrazioni effettuate, ecc.) & 
        \\\hline

%% journal
% ------------------------------------------------------------------------------
        
        \iffalse
        \journalentry{Giorno}{Data}{Ore}{
            Descrizione
        }{
            Osservazioni
        }
        \fi

        \journalentry{Gio}{12.04.2018}{2}{
            Preparazione della documentazione, della pianifica ed organizzazione
            generale del progetto.
        }{}

        \journalentry{Gio}{12.04.2018}{5}{
        	Analisi e studio del concetto matematico.
        }{}

        \journalentry{Gio}{12.04.2018}{1}{
        	Raccolto informazioni e librerie software per i componenti utilizzati
        	dal progetto. Inoltre \`e stata preparata una struttura per la
        	documentazione.
        }{}

        \journalentry{Ve}{13.04.2018}{2}{
        	Scelto i componenti analogici e passivi e preparato una BOM (Bill Of 
        	Materials). Progettato uno schema elettrico.

        }{
   			\`E stato scelto di utilizzare un unico amplificatore di qualit\`a 
   			migliore (per audio) con un multiplexer invece di un package con pi\`u
   			amplificatori di precisione inferiore.
        }

        \journalentry{Ve}{13.04.2018}{3}{
        	Realizzato la parte centrale dello schema elettrico in formato ECAD
        	(Altium Designer). Ossia i circuiti di multiplexing, di adattamento
        	del segnale e di filtraggio delle frequenze indesiderate.
        }{
       		Manca il jack di alimentazione (non ancora necessaria su tavola 
       		sperimentale) e il circuito di adattamento di tensione da 12\,V a 5\,V.
        }

        \journalentry{Ve}{13.04.2018}{2}{
        	Modificato il circuito progettato per utilizzare un filtro attivo
        	anzich\`e passivo dopo aver osservato sperimentalmente
        	l'attenuazione dal filtro passivo.
        }{
        	Il circuto su piastra sperimentale non \`e ancora funzionante
        	per sviluppare i primi programmi di test.
        }

        \journalentry{Ve}{13.04.2018}{2}{
        	Studiato il concetto matematico della Fourier transform con il 
        	professor Eduardo Cima.
        }{
        	L'attivit\`a non era programmata poich\`e la disponibilit\`a
        	dei docenti \`e limitata.
        }

% ------------------------------------------------------------------------------
    \end{tabularx}
\end{document}
