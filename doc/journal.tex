\documentclass[a4paper]{article}

% metadata
% ------------------------------------------------------------------------------

\newcommand{\name}{Naoki Pross}
\newcommand{\instructor}{Rinaldo Geiler, Daniele Kamm}
\newcommand{\project}{Spectrum Analyzer}
\newcommand{\projstart}{12.04.2018}
\newcommand{\projend}{15.04.2018}

% auto
\newcommand{\projperiod}{\projstart{} -- \projend{}}

% preable
% ------------------------------------------------------------------------------

% tables
\usepackage{tabularx}
\usepackage{booktabs}
\usepackage{array}
\usepackage{multirow}
\usepackage{ltablex}
\keepXColumns

% colors
\usepackage[table]{xcolor}
\usepackage{graphicx}

% font
% \usepackage[sfdefault]{noto}
% \usepackage{lmodern}
\renewcommand{\familydefault}{\sfdefault}

% layout
\usepackage[left=2cm, right=2cm, top=3cm, bottom=3cm]{geometry}
\usepackage{fancyhdr}

% header and footer
\pagestyle{fancy}
\fancyhead[L]{CAM-SAM}
\fancyhead[C]{Elettronico}
\fancyhead[R]{22.02.2018}
\fancyfoot[L]{\jobname.tex}
\fancyfoot[C]{\name}
\fancyfoot[R]{\thepage}

\renewcommand\arraystretch{1.5}
\renewcommand\tabcolsep{5pt}
\setlength{\parindent}{0pt}
\setlength{\parskip}{1em}

% macros
\newcommand{\journalentry}[5]{%
    #1 & #2 & #3 & #4 & #5 \\\hline
}

% document
% ------------------------------------------------------------------------------

\begin{document}
    \begin{center}
        \bf \LARGE DIARIO GIORNALIERO
    \end{center}
    
    \vspace{5mm}

    \begin{tabularx}{\textwidth}{| lX | lX |}
        \hline
        \bfseries Candidato: & \name &
        \bfseries Progetto: & \project
        \\\hline
        \bfseries Formatore: & \instructor &
        \bfseries Periodo: & \projperiod
        \\\hline
    \end{tabularx}
   
    \vspace{5mm}
    
    \begin{tabularx}{\textwidth}{| c | c | c | p{.4\textwidth} | X |}
        \hline    
        \rowcolor{gray!30}
        \bfseries Giorno &
        \bfseries Data & 
        \bfseries Ore &
        \bfseries Descrizione attivit\`a &
        \bfseries Osservazioni
        \\
        \rowcolor{gray!30}
        & & &
        (Attivit\`a eseguite, metodi adottati, decisioni prese,
        dimostrazioni effettuate, ecc.) & 
        \\\hline

%% journal
% ------------------------------------------------------------------------------
        
        \iffalse
        \journalentry{Giorno}{Data}{Ore}{
            Descrizione
        }{
            Osservazioni
        }
        \fi

        \journalentry{Gio}{12.04.2018}{2}{
            Preparazione della documentazione, della pianifica ed organizzazione
            generale del progetto
        }{}
        
% ------------------------------------------------------------------------------
    \end{tabularx}
\end{document}
